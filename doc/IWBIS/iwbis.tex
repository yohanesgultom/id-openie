
%% bare_conf_compsoc.tex
%% V1.4b
%% 2015/08/26
%% by Michael Shell
%% See:
%% http://www.michaelshell.org/
%% for current contact information.
%%
%% This is a skeleton file demonstrating the use of IEEEtran.cls
%% (requires IEEEtran.cls version 1.8b or later) with an IEEE Computer
%% Society conference paper.
%%
%% Support sites:
%% http://www.michaelshell.org/tex/ieeetran/
%% http://www.ctan.org/pkg/ieeetran
%% and
%% http://www.ieee.org/

%%*************************************************************************
%% Legal Notice:
%% This code is offered as-is without any warranty either expressed or
%% implied; without even the implied warranty of MERCHANTABILITY or
%% FITNESS FOR A PARTICULAR PURPOSE! 
%% User assumes all risk.
%% In no event shall the IEEE or any contributor to this code be liable for
%% any damages or losses, including, but not limited to, incidental,
%% consequential, or any other damages, resulting from the use or misuse
%% of any information contained here.
%%
%% All comments are the opinions of their respective authors and are not
%% necessarily endorsed by the IEEE.
%%
%% This work is distributed under the LaTeX Project Public License (LPPL)
%% ( http://www.latex-project.org/ ) version 1.3, and may be freely used,
%% distributed and modified. A copy of the LPPL, version 1.3, is included
%% in the base LaTeX documentation of all distributions of LaTeX released
%% 2003/12/01 or later.
%% Retain all contribution notices and credits.
%% ** Modified files should be clearly indicated as such, including  **
%% ** renaming them and changing author support contact information. **
%%*************************************************************************


\documentclass[conference,compsoc]{IEEEtran}

% *** CITATION PACKAGES ***
\usepackage[nocompress]{cite}

% *** GRAPHICS RELATED PACKAGES ***
%
\usepackage[pdftex]{graphicx}
% declare the path(s) where your graphic files are
\graphicspath{{../images/}}
% and their extensions so you won't have to specify these with
% every instance of \includegraphics
\DeclareGraphicsExtensions{.jpeg,.png}

% *** MATH PACKAGES ***
\usepackage{amsmath}
% Note that the amsmath package sets \interdisplaylinepenalty to 10000
% thus preventing page breaks from occurring within multiline equations. Use:
%\interdisplaylinepenalty=2500

% *** ALIGNMENT PACKAGES ***
\usepackage{array}

% *** SUBFIGURE PACKAGES ***
\usepackage[caption=false,font=footnotesize,labelfont=sf,textfont=sf]{subfig}

% *** FLOAT PACKAGES ***
%\usepackage{fixltx2e}
%\usepackage{stfloats}
% \usepackage{dblfloatfix}

% *** PDF, URL AND HYPERLINK PACKAGES ***
\usepackage{url}
\usepackage{bm}
%\usepackage{algorithm}
%\usepackage{algorithmic}

% *** Temporary packages ***
\usepackage{lipsum}
\usepackage{pifont}
%\usepackage{xcolor}
%\newcommand\mytodo[1]{\textcolor{red}{#1}}

% correct bad hyphenation here
\hyphenation{op-tical net-works semi-conduc-tor}

\begin{document}

% paper title
\title{Automatic Open Domain Information Extraction\\from Indonesian Text}


% author names and affiliations
% use a multiple column layout for up to three different
% affiliations
\author{
	\IEEEauthorblockN{Yohanes Gultom}
	\IEEEauthorblockA{Faculty of Computer Science\\
	University of Indonesia\\
	Email: yohanes.gultom@ui.ac.id}
	\and
	\IEEEauthorblockN{Wahyu Catur Wibowo}
	\IEEEauthorblockA{Faculty of Computer Science\\
	University of Indonesia\\
	Email: wibowo@cs.ui.ac.id}
}


% make the title area
\maketitle

% As a general rule, do not put math, special symbols or citations
% in the abstract
\begin{abstract}

Availability of big amount digital documents calls for an automatic method to extract information from any text document regardless of domain. Unfortunately, existing open domain information extraction (open IE) systems are not suitable for low-resource language such as Indonesian. This paper introduces a system to extract relation triples from Indonesian text using rule-based triple candidates generator, rule-based token expander and machine-learning-based triple selector. Trained using our 2,344 triples dataset (166 positives \& 2,183 negatives), a Random Forest triple selector model achieves cross-validation score of 0.58 F1 (0.62 precision and 0.58 recall).

\end{abstract}


\section{Introduction}

Open domain information extraction (open IE) is a paradigm that facilitates domain-independent discovery of triple relations from text document\cite{banko2007open}. It extracts relations from sentence in three-values tuples or triples format $(x, r, y)$ where $x$ and $y$ called arguments and $r$ is the relation\cite{etzioni2011open}. In more linguistic term, the arguments are also referred as subject and object while relation are referred as predicate\cite{angeli2015leveraging}. The example of this extraction is described in Figure \ref{fig_example_io_openie}.

As described in Table \ref{table_paradigm_comparison}, unlike traditional information extraction (IE), open IE extracts domain-independent relations from sentence. While it retrieves relations in format of triples similar to knowledge extraction (KE), open IE doesn't follow whole Resource Data Format (RDF) specification\footnote{\url{https://www.w3.org/RDF/}} like KE\cite{auer2007dbpedia} \cite{exner2014refractive}. Although mapping to existing relation schema is required in real word task such as slot filling\cite{angeli2015leveraging}, ontology is not in the scope of open IE research. Open IE has also been reported to be useful for tasks such as question answering\cite{fader2011identifying} and information retrieval\cite{etzioni2011search}. 

\begin{table}[!t]
\renewcommand{\arraystretch}{1.5}
\caption{General comparison between traditional information, open domain information and knowledge extraction}
\label{table_paradigm_comparison}
\centering
\begin{tabular}{|l|>{\centering\arraybackslash}p{1.5cm}|>{\centering\arraybackslash}p{1.5cm}|>{\centering\arraybackslash}p{1.5cm}|}
\hline 
 & \textbf{IE} & \textbf{Open IE} & \textbf{KE} \\ 
\hline 
\textbf{Domain} & Closed & Open & Open \\ 
\hline 
\textbf{Format} & Depends on domain & Triples & RDF Triples \\ 
\hline 
\textbf{Ontology} & Not available & Optional & Mandatory \\ 
\hline 
\end{tabular} 
\end{table}

\begin{figure}
\textbf{Input} \\[0.1cm]
"Sembungan adalah sebuah desa yang terletak di kecamatan Kejajar, kabupaten Wonosobo, Jawa Tengah, Indonesia." \\[0.5cm]
\textbf{Output} \\[0.1cm]
1. (Sembungan, adalah, desa) \\
2. (Sembungan, terletak di, kecamatan Kejajar) \\
\caption{Example of expected input and output of open domain information extraction}
\label{fig_example_io_openie}
\end{figure}

Due to the nature of NLP tasks and heuristics used in open IE system, it is only applicable for a specific language\cite{banko2007open}. So in order to extract open domain information from Indonesian text, a specific system has to be defined for this language. Furthermore, considering the scarcity of Indonesian NLP resources, the system need to effectively utilize them to achieve the objective. Through this paper, we propose a open IE system that addresses these issues.

We propose an open IE system that combine heuristics (rule-based) models and a supervised learning model to extract relation triples from Indonesian text. This approach only requires single manually annotated dataset which is required to train triple selector/classifier. Our objective is to define a baseline system for Indonesian open IE that may be encourage more research in the future.

In general, the contributions from this research are:

\begin{itemize}
\item Open domain information extraction system for Indonesian text
\item Open-source implementation of the system in public repository\footnote{\url{https://github.com/yohanesgultom/id-openie}}
\item Dataset of manually tagged triple candidates
\item Reusable Indonesian NLP pipelines (lemmatizer, part of speech tagger, named-entity recognizer and dependency parser) built by extending Stanford CoreNLP\footnote{\url{https://stanfordnlp.github.io/CoreNLP}} API
\end{itemize}

Further in this paper we will described some of the preeminent related works in open IE, the details about proposed system, experiments using some supervised-learning models as triples selector, analysis of the experiments results, and finally, conclusions and future works of this research.

\section{Related Work}

There has been plenty of works done in the open IE research. Starting from the introduction of open IE along with its first fully-implemented system, TextRunner, which further succeeded by systems built on top of it: ReVerb, R2A2 and Ollie (all from the same research group). The most recent research introduces Stanford OpenIE\footnote{\url{https://nlp.stanford.edu/software/openie.html}} which is an implementation of novel open IE system that outperforms Ollie in TAC-KBP
2013 Slot Filling task\cite{angeli2015leveraging}.

TextRunner was designed for massive size of open-domain web documents by avoiding heavy linguistic tasks and used inverted index to store extraction result\cite{banko2007open}. It generates its own dataset (self-supervised) by using part of speech and dependency features and train a naive bayes classifier to select the triples. It argues that heavy linguistic tasks such as dependency parsing are not scalable to handle million of web documents. Additionally, it also uses redundancy assessor to remove redundant words (stop words, adverbs .etc).

ReVerb is an immediate successor of TextRunner which solves two significant problems in its predecessor: incoherent extractions and uninformative extractions \cite{fader2011identifying}. It is composed of two algorithm: (1) Relation Extraction that extracts relations using syntactical and lexical constraint to solve the problems, and (2) Argument Extraction which retrieve the noun phrases as arguments of the relation. ReVerb takes as input a POS-tagged and NP-chunked and returns a set of relation triples.

R2A2 is a system built to fix argument extraction problem in ReVerb \cite{etzioni2011open}. Instead of using heuristics to extract the arguments, it uses a learning-based system, ArgLearner, that accepts relation and sentence as inputs and returns the first (Arg1) and second arguments (Arg2). ArgLearner extracts the arguments using three classifiers based on REPTree and sequence labeling CRF as described in Figure \ref{fig_arglearner_architecture}.

\begin{figure}
\centering
\includegraphics[scale=0.5]{arglearner_architecture}
\caption{ArgLearner architecture training and extraction architecture}
\label{fig_arglearner_architecture}
\end{figure}

Furthermore, Ollie (Open Language Learning for Information Extraction)\cite{schmitz2012open} utilizes ReVerb to learn open pattern templates to guide triples extraction from sentence. Additionally, Ollie does a context analysis to extend the tuples with contextual information in order to improve precision. Its training and extraction architecture is describe in Figure \ref{fig_ollie_architecture}.

\begin{figure}
\centering
\includegraphics[scale=0.5]{ollie_architecture}
\caption{Ollie labeling and extraction architecture}
\label{fig_ollie_architecture}
\end{figure}

One of the most research proposes new open IE system that replaces the usage of large open patterns in Ollie with a set of fewer patterns for canonically structured sentences and a classifier that learns to extract self-contained clauses from a sentence\cite{angeli2015leveraging}. This system is implemented in Stanford OpenIE which is also integrated in the populer open source suites, Stanford Core NLP.

\section{Proposed System}

Our proposed system also follows the pattern of three-steps\cite{etzioni2011open} method used by open IE system:

\begin{enumerate}

\item \textbf{Label}: sentence are labeled to create a training dataset for the classifier. Although most of the related systems choose to do it automatically (using heuristics or distant supervision)\cite{banko2007open}\cite{etzioni2011open}\cite\cite{schmitz2012open}, we choose to follow the method in recent research\cite{angeli2015leveraging} to manually label our training data to ensure the quality.

\item \textbf{Learn}: train a classifier using the dataset to extract dataset. We use an ensemble model, Random Forest\cite{breiman2001random}, as a classifier since it achieves the best score in our experiment.

\item \textbf{Extract}: use the classifier to extract relations (predicate) and arguments (subjects \& objects) as triples. In our case, we also do token expansion to expand the token into meaningful clause.

\end{enumerate}

As shown in the flowchart Figure \ref{fig_program_flowchart}, our system has three main components:

\begin{figure*}[!t]
\centering
\includegraphics[width=\textwidth]{program_flowchart}
\caption{Indonesian open domain information extraction flowchart}
\label{fig_program_flowchart}
\end{figure*}

\subsection{NLP Pipeline}

The NLP pipeline is a series of NLP tasks that annotates one or more sentences and saves them in CONLL-U\footnote{\url{http://universaldependencies.org/format.html}} format, a token-based sentence annotation format containing lemma, POS tag, dependency relation and a slot for additional annotation. The pipeline assumes that each sentence in the input document is separated by new line so preprocessing may be required. The detail of each model the pipeline are described below:

\begin{enumerate}

\item \textbf{Tokenizer} \\
We use default tokenizer provided by Stanford Core NLP, \verb|PTBTokenizer|\cite{manningptbtokenizer}, which mimics Penn Treebank 3 tokenizer\footnote{\url{https://catalog.ldc.upenn.edu/LDC99T42}}. While this tokenizer provides many options to modify its behavior, we stick to default configuration that split sentence by whitelines to get the tokens.\\

\item \textbf{Part of Speech Tagger} \\
We trained default Stanford Core NLP \verb|MaxentTagger|\cite{toutanova2003feature} with Indonesian universal POS tag dataset which we convert from dependency parsing dataset\footnote{\url{https://github.com/UniversalDependencies/UD_Indonesian}}. This POS tagger uses Max Entropy (multi-class logistic regression) classifier which yields \textbf{93.68\%} token accuracy and \textbf{63.91\%} sentence accuracy when trained using 5,036 sentences and tested with 559 sentences from the dataset. \\

\item \textbf{Lemmatizer} \\
The lemmatizer used in this pipeline, \verb|IndonesianLemmaAnnotator|, is implemented based on an existing Indonesian rule-based Lemmatizer\cite{suhartono2014lemmatization} with some improvements:

\begin{itemize}
\item Reimplementation in Java language
\item Usage of in-memory database to speed up dictionary lookup
\item Integration with Stanford Core NLP annotator API for reusability
\end{itemize}

This lemmatizer yields \textbf{99\%} accuracy when tested using dataset of 5,638 token-lemma pairs\footnote{\url{https://github.com/davidchristiandy/lemmatizer}}. \\

\item \textbf{Named-Entity Recognizer} \\

Stanford NLP \verb|CRFClassifier|\cite{finkel2005incorporating}, a linear chain Conditional Random Field (CRF) sequence models, is trained using a dataset containing 3,535 Indonesian sentences with 5 entity class: Person, Organization, Location, Quantity and Time. When tested using 426 sentences, this models achieves 0.86 precision, 0.85 recall and \textbf{0.86} F1-score. The dataset itself is a combination between dataset from Faculty of Computer Science, University of Indonesia and a public dataset\footnote{\url{https://github.com/yusufsyaifudin/indonesia-ner}}. \\

\item \textbf{Dependency Parser} \\

We relied on Stanford NLP \verb|nndep.DependencyParser|\cite{chen2014fast}, to annotate dependency relation of each token in the the sentence. We train this transition-based neural network model using a Indonesian universal dependencies dataset of 5,036 sentences and 3,093 Indonesian word embeddings\footnote{\url{https://github.com/yohanesgultom/id-openie/blob/master/data/parser-id.embed}} (vector representation of words). Tested with 559 sentences, this model scores \textbf{70\%} UAS (Unlabeled Attachment Score) and \textbf{46\%} LAS (Labeled Attachment Score).


\end{enumerate}

\begin{figure}
\centering
\includegraphics[scale=0.35]{conllu_example}
\caption{CONLL-U Format Example}
\label{fig_conllu_example}
\end{figure}

This pipeline is built by extending Stanford Core NLP classes and packaged as single Java program (JAR). Therefore it can be reused in any other system that require same kind of annotations.

\subsection{Triple Candidates Generator} \label{Triple Candidates Generator}

% Similar to TextRunner Self-Supervised Learner but doesn't automatically label triples

Triple candidates generator is used to extract relation triples candidates from CONLL-U document produced by NLP pipeline. It uses a set of rules listed in Table \ref{table_triple_candidate_generation_rules} to extract relations (predicates) and arguments (subjects and predicates) from the sentence. The results of triples extraction are not always the positive or valid relation triples so, unlike TextRunner\cite{banko2007open}, we cannot use them directly as training data for triple selector/classifier.

For example, applying the rules to a annotated sentence in Figure \ref{fig_conllu_example} will generate these 17 triples candidates--where only five of them are valid triples:

\begin{itemize}
\item (Sembungan, adalah, desa) \ding{51}
\item (Sembungan, adalah, terletak)
\item (Sembungan, adalah, kecamatan)
\item (Sembungan, adalah, kabupaten)
\item (Sembungan, adalah, Jawa)
\item (Sembungan, adalah, Tengah)
\item (Sembungan, adalah, Indonesia)
\item (Sembungan, terletak, kecamatan) \ding{51}
\item (Sembungan, terletak, kabupaten) \ding{51}
\item (Sembungan, terletak, Jawa) \ding{51}
\item (Sembungan, terletak, Tengah)
\item (Sembungan, terletak, Indonesia) \ding{51}
\item (desa, terletak, kecamatan)
\item (desa, terletak, kabupaten)
\item (desa, terletak, Jawa)
\item (desa, terletak, Tengah)
\item (desa, terletak, Indonesia)
\end{itemize}

In order to build a training data for the triple selector, we used triple candidates generator to generate 1,611 triple candidates from 42 sentences. As part of the label step, we manually labeled \textbf{132 positive} and \textbf{1,479 negative} triples which we use to train binary classifier as triple selector in the learn step.

During the extraction step, triple candidates generator is used in the system to extract unlabeled candidates from CONLL-U document. These unlabeled triples will be labeled by trained triple selector as described in  (referring to flowchart in Figure \ref{fig_program_flowchart}.

% Triple candidate generation rules
\begin{table}[!t]
\renewcommand{\arraystretch}{1.5}
\caption{Triple candidate generation rules}
\label{table_triple_candidate_generation_rules}
\centering
\begin{tabular}{l|p{6cm}}
\hline
\textbf{Type} & \textbf{Condition} \\
\hline
Subject & Token's POS tag is either PROPN, NOUN, PRON or VERB \\
\space & Token is not "yang" nor "adalah" \\
\space & Token's dependency is neither "compound" nor "name" \\
\space & Token's dependency is either "compound" or "name" but separated by more than 2 tokens from its head \\
\hline
Predicate & Token's position is after Subject \\
\space & Token's POS tag is either VERB or AUX \\
\hline
Object & Token's position is after Subject and Predicate \\
\space & Token's POS tag is either PROPN, NOUN, PRON or VERB \\
\space & Token is not "yang" nor "adalah" \\
\space & Token's dependency is neither "compound" nor "name" \\
\space & Token's dependency is either "compound" or "name" but separated by more than 2 tokens from its head \\
\end{tabular}
\end{table}


\subsection{Triple Selector}  \label{Triple Selector}

Triple selector is a machine learning classifier trained using manually labeled dataset of valid and invalid relation triples. For example, given the input of 17 candidates in Section \ref{Triple Candidates Generator}, the selector will label the five check-marked triples as true and label the rest as false.

We use Random Forest\cite{breiman2001random}, an ensemble methods that aggregate classification results from multiple decision trees, as the model for the classifier. We use the Scikit-Learn\footnote{\url{http://scikit-learn.org}} implementation of Random Forest with following configuration:

\begin{itemize}
\item Decision tree criterion: Gini Impurity
\item Minimum number of samples to split internal node: 5
\item Maximum trees depth: 8
\item Number of trees: 20
\item Maximum features used in each tree: 4 (square root of the number of features)
\item Class weight: balanced (multiplied by the ratio of training samples)
\end{itemize}

By using Grid Search\cite{wasserman2015grid}, an exhaustive search algorithm to find optimal hyper-parameters, we discover that this configuration yields the best F1 score for Random Forest classifier when trained and tested with our dataset described in Section \ref{Triple Candidates Generator}. 

We extract 18 features described in Table \ref{table_models_features} from each triple candidates. Our system uses features based on POS tag, named-entity and dependency relation, instead of shallow syntactic features used by TextRunner or ReVerb\cite{banko2007open}\cite{etzioni2011open}. Every nominal features are encoded and normalized along with the whole dataset by removing the mean and scaling to unit variance in order to improve the precision and recall of the classifier.

\begin{table}[!t]
\renewcommand{\arraystretch}{1.5}
\caption{Triple selector features}
\label{table_models_features}
\centering
\begin{tabular}{r|l}
\hline
\textbf{\#} & \textbf{Features} \\
\hline
1 & Subject token's POS tag \\
2 & Subject token's dependency relation \\
3 & Subject token's head POS tag \\
4 & Subject token's named entity \\
5 & Subject token's distance from predicate \\
7 & Subject token's dependency with predicate \\
8 & Predicate token's POS tag \\
9 & Predicate token's dependency relation \\
10 & Predicate token's head POS tag \\
11 & Predicate token's dependents count \\
12 & Object token's POS tag \\
13 & Object token's dependency relation \\
14 & Object token's head POS tag \\
15 & Object token's named entity \\
16 & Object token's dependents count \\
17 & Object token's distance from predicate \\
18 & Object token's dependency with predicate \\
\end{tabular}
\end{table}

During the train step, we use the dataset prepared in the label step (described in Section \ref{Triple Candidates Generator}) to train triple selector and save the best model as binary file. This model is included in the system to be use during the extraction step.

\subsection{Token Expander}

Instead of using lightweight noun phrase chunker\cite{banko2007open}, our system uses rule-based token expander to extract relation or argument clauses. It uses heuristics based on syntactical features (POS tag, dependency relation and named-entity) described in Table \ref{table_token_expansion_rules_s_o} and Table \ref{table_token_expansion_rules_p} to determine whether to expand a token to its dependent, ignore the dependent or even remove the token itself. For example, token expander will expand check-marked triples in Section \ref{Triple Candidates Generator} into:

\begin{itemize}
\item (Sembungan, adalah, desa)
\item (Sembungan, terletak di, kecamatan Kejajar)
\item (Sembungan, terletak di, kabupaten Wonosobo)
\item (Sembungan, terletak di, Jawa Tengah)
\item (Sembungan, terletak di, Indonesia)
\end{itemize}

% Token expansion rules for Subject or Object token
\begin{table}[!t]
\renewcommand{\arraystretch}{1.5}
\caption{Token expansion rules for Subject or Object token}
\label{table_token_expansion_rules_s_o}
\centering
\begin{tabular}{r|p{6cm}|l}
\hline
\textbf{\#} & \textbf{Condition for Subject or Object Token} & \textbf{Action} \\
\hline
1 & If dependent's relation to the token  is either “compound”, “name”  or “amod” & Expand \\
2 & If dependent has same named entity as the token & Expand \\
3 & If dependent and the token are wrapped by quotes or double quotes  & Expand \\
4 & If the head is a sentence root & Ignore \\
5 & If dependent's POS tag is CONJ or its form is either “,” (comma) or “/” (slash) & Ignore \\
6 & If dependent's POS tag is either “VERB” or “ADP” & Ignore \\
7 & If dependent has at least one dependent with “ADP” POS tag & Ignore \\
8 & If the first or last token in expansion result has “CONJ” or “ADP” POS tag & Remove \\
9 & If the first or last index of expansion result is an incomplete parentheses symbol & Remove \\
10 & If the last index of expansion result is “yang” & Remove \\
11 & Else & Ignore \\

\end{tabular}
\end{table}

\begin{table}[!t]
\renewcommand{\arraystretch}{1.5}
\caption{Token expansion rules for Predicate token}
\label{table_token_expansion_rules_p}
\centering
\begin{tabular}{r|p{6cm}|l}
\hline
\textbf{\#} & \textbf{Condition for Predicate Token} & \textbf{Action} \\
\hline
1 & If dependent is “tidak” & Expand \\
2 & Else & Ignore \\
\end{tabular}
\end{table}

During the label step, token expander is used to make manual annotation process easier. We label a triple candidate as valid only if it makes sense after being expanded to clause. For example, \textit{(Sembungan, terletak, kecamatan)} doesn't really make sense before expanded to \textit{(Sembungan, terletak di, kecamatan Kejajar)}.

\section{Experiments} \label{Experiments}

In this research, we report two experiments. The first one shows the performance comparison of four classifiers in selecting valid triples from given candidates. While the second one shows the scalability of our system (using the best classifier) extracting triples from documents (unannotated). Both of the experiments are run on an Ubuntu 15.04 64-bit, Intel Core i7 5500U (dual cores), DDR3 8 GB RAM, SSD 250 GB machine.

In the first experiment, we chose four classifiers each representing unique characteristics: 

\begin{enumerate}
\item Linear Logistic Regression\cite{fan2008liblinear} (linear model)
\item Polynomial Support Vector Machine (SVM)\cite{chang2011libsvm} (nonlinear model)
\item Multi Layer Perceptron (MLP)\cite{hinton1989connectionist} with 2 hidden layers (20 and 10 ReLU\cite{nair2010rectified} neurons)
\item Random Forest\cite{wasserman2015grid} (ensemble decision trees)
\end{enumerate}
  
We use the manually annotated triple selector dataset described in Section \ref{Triple Candidates Generator} to cross-validate\cite{kohavi1995study} (k-Fold with k = 3) the four classifiers. Since open IE systems requires both precision and recall\cite{angeli2015leveraging}, we choose F1 score to determine the best classifier for triple selector. The result of this experiment is shown by Figure \ref{fig_models_performance} and Table \ref{table_models_performance} where Random Forest achieves the highest F1 score 0.58.

\begin{figure}
\includegraphics[scale=0.4]{models_performance}
\caption{Triple selector models performance comparison chart}
\label{fig_models_performance}
\end{figure}

\begin{table}[!t]
\renewcommand{\arraystretch}{1.5}
\caption{Triple selector models performance}
\label{table_models_performance}
\centering
\begin{tabular}{l r r r}
\hline
\textbf{Models} & \textbf{P} & \textbf{R} & $\mathbf{F_1}$ \\
\hline
Logistic Regression & 0.64 & 0.28 & 0.36 \\
SVM & \textbf{0.68} & 0.41 & 0.51 \\
MLP & 0.54 & 0.46 & 0.47 \\
Random Forest & 0.62 & \textbf{0.58} & \textbf{0.58} \\
\hline
\end{tabular}
\end{table}

In the second experiment, we evaluate the performance of our system by extracting triples from three documents with different number of sentences, measure the total execution time and calculate the average execution time per sentence. The result in Table \ref{table_system_extraction_time} shows that the lowest execution time (or fastest execution time) is 0.014 seconds when processing document of 5,593 sentences.

\begin{table}[!t]
	\renewcommand{\arraystretch}{1.5}
	\caption{System end-to-end extraction time}
	\label{table_system_extraction_time}
	\centering
	\begin{tabular}{l p{1.2cm} p{1.2cm} p{1.2cm}}
		\hline
		\textbf{Sentences} & \textbf{Triples Extracted} & \textbf{Total Time (s)} & \textbf{Time per Sentence (s)} \\
		\hline
		2 & 7 & 6.1 & 0.800 \\
		138 & 429 & 11.3 & 0.082 \\
		5,593 & 19,403 & 78.6 & 0.014 \\
		\hline
	\end{tabular}
\end{table}

\section{Analysis}

As implied by the flowchart in Figure \ref{fig_program_flowchart}, the performance of our system depends on the performance of NLP pipeline. Since most of the heuristics and features used in our systems are based on POS tag and dependency relation, the low performance of the models responsible for these tasks are reducing the overall performance.

The heuristics used in triple candidate generator and token expander are still limited to explicit pattern. For instance, triple candidate generator can not extract relations \textit{(kecamatan Kejajar, terletak di, Jawa Tengah)} and \textit{(Jawa Tengah, terletak di, Indonesia)} from the sentence in Figure \ref{fig_example_io_openie} yet. While the token expander is having problem in expanding token to implicitly expected clauses such as \textit{"seorang pelatih sepak bola"} from \textit{"seorang pelatih dan pemain sepak bola"} or \textit{"satu buah torpedo"} from \textit{"satu atau dua buah torpedo"}.

The first experiment shows that all classifiers are still having problem learning the pattern of triples. We suspect this is primarily caused by insufficient and imbalanced training data. The low performance of linear logistic regression also shows that this problem is not linearly separable. The random forest performs better than other nonlinear models because it is easily tuned to balance the precision and recall by changing the depth of decision trees.

Speed-wise, our system shows reasonable average performance of 0.014 seconds per sentence (for 5,593 sentences) where TextRunner runs on average 0.036 CPU seconds per sentence\cite{banko2007open}. While the speed improvement for bigger document shows that the "heavy" linguistic tasks such as dependency parsing doesn't cause performance drawback in big document.

\section{Conclusion}

This paper introduces an open domain information extraction system for Indonesian text using basic NLP pipelines and combination of heuristics and machine learning models. The system is able to extract meaningful domain-independent relations from Indonesian sentences to be used as document representation or document understanding task. Additionally, the source code and datasets are published openly\footnote{\url{github.com/yohanesgultom/id-openie}} to improve research reproducibility.

In the future, we plan to improve the performance by adding more training data for triple selector and finding better heuristics in triple candidate generator to reduce the negative samples. We also plan to use create additional dataset and use machine learning model for token expansion to improve the result. Adding confidence level in the output of every phases (NLP pipelines, candidate generator, triple selector, token expander) and include it as features and heuristics may also improve the overall performance of the system.

% conference papers do not normally have an appendix

% use section* for acknowledgment
%\ifCLASSOPTIONcompsoc
  % The Computer Society usually uses the plural form
%  \section*{Acknowledgments}
%\else
  % regular IEEE prefers the singular form
%  \section*{Acknowledgment}
%\fi

\bibliographystyle{IEEEtran}
\bibliography{../pustaka}

% that's all folks
\end{document}
